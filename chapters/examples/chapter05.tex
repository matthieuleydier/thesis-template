\chapter{Conclusion}
\label{ch:Conclusion}

This work has been dedicated to characterise the three FBK detectors. Methods already developed in \cite{Girard2018CharacterisationDistributions} have been reproduced with the Hamamatsu H2017 and applied to the FBK SiPMs. \\
Three different experiment setups were employed and used to characterise different properties of the \SI{16}{\micro m}, \SI{31}{\micro m}, \SI{42}{\micro m} SiPM from FBK and the H2017 SiPM from Hamamatsu.
The IV setup provided measurements of the breakdown voltage less accurate than the waveform analysis and was abandoned. Nevertheless, the quenching resistor $R_Q$ has been measured for $16$ channels out of $128$ with acceptable results. A mean $R_Q$ of \SI{550}{\kilo \ohm}, \SI{436}{\kilo \ohm}, \SI{444}{\kilo \ohm}  and \SI{503}{\kilo \ohm} for H2017, \SI{16}{\micro m}, \SI{31}{\micro m} and \SI{42}{\micro m} respectively compared to an announced value of  \SI{500}{\kilo \ohm} for all was measured. The spread is below $5\%$ for the FBK SiPMs and of $7.4\%$ for the Hamamatsu. 
\\

The waveform analysis setup has enabled us to get breakdown voltage values with an accuracy of  \SI{200}{\milli \volt} for all detectors except the FBK \SI{16}{\micro m} (\SI{300}{\milli \volt}). Two channels per detector were analysed and the results were compared with the values with the VATA64 method.  
The correlated noise has also been obtained with the waveform analysis. FBK detectors can operate up to $\Delta V = 12$ V whereas the Hamamatsu is limited to $ \Delta V = 5$V. This increase of operation range is possible because the correlated noise for the FBK detectors is really low:  $(3.8 \pm 0.2)\%$, $(4.6\pm0.3)\%$ and $(5.2\pm0.3)\%$ at $\Delta V = 12$ V for the \SI{16}{\micro m}, \SI{31}{\micro m} and \SI{42}{\micro m} respectively. At $\Delta V = 4$ V the Hamamatsu H2017 has $(7.5\pm 0.3)\%$ of correlated noise which represents $9.3$ times more noise than the FBK \SI{42}{\micro m}. 
%Nevertheless, due to a too little signal at very low overvoltage, the SiPMs with the smallest pixels from FBK cannot operate well below $5 \Delta V$.
\\
%The difference in operation range and correlated noise is very important, 
The PDE and Gain setup coupled to the wave analysis allowed to measure the gain and PDE for each detector. A comparison of the \SI{31}{\micro m} and another \SI{31}{\micro m} with an additional layer of epoxy glue was studied to see its effect on the PDE. 
The gain of the FBK detectors has been measured with $2\%$ errors and values of $(0.142 \pm 3)\cdot10^{6} $V$^{-1}$, $(0.446\pm 0.009)\cdot10^{6} $V$^{-1}$ and $(0.770\pm 0.015)\cdot10^{6} 
$V$^{-1}$ for the \SI{16}{\micro m}, \SI{31}{\micro m} and \SI{42}{\micro m} respectively. The Hamamatsu H2017 gain was the same value as measured in \cite{Girard2018CharacterisationDistributions}, $(1.022\pm 0.02)\cdot10^{6}$V$^{-1}$. These measurements showed the relation between the pixel size and the gain. They are also a way to cross check the $V_{bd}$ value. 
The PDE measurements showed and oscillating behavior of the PDE as the wavelength varied. Reaching a maximum around \SI{410}{\nano m} of $29\%$, $52\%$, $57\%$ for \SI{16}{\micro m} , \SI{31}{\micro m} and \SI{42}{\micro m} at $\Delta V = 12$ V (for the optimal $\Delta V$ and with a $\approx 4\%$ amplitude). The Hamamatsu has at $\lambda \approx 470$ nm a peak PDE of $40\%$ at $\Delta V = 4.4$ V.
Adding an epoxy layer on the FBK \SI{31}{\micro m} cancelled the oscillations and reduced the PDE by $10\%$ of the \SI{31}{\micro m} bare die with a peak efficiency at measured $46\%$ at $\Delta V=12$ V.

In conclusion, the FBK detectors have proven to be really performing at high overvoltage due to their low correlated noise. This high operating range open possibilities for higher PDE. 
A baseline of important characteristics of the FBK SiPMs has been done. For the LHCb upgrade, other measurements should be investigated in order to decide whether these are convenient or not. The NUV-HD technology is designed to have optimal performances at cryogenic temperatures, a study on the reduction of DCR at low temperatures could be done. The LHC upgrade will have more luminosity implying more radiation damage on the SiPMs. Studies about the impact of radiation on the FBK NUV-HD SiPMs are needed. 
