%*******************************************************
% Abstract in English
%*******************************************************
\pdfbookmark[0]{Abstract}{Abstract}


\begin{otherlanguage}{american}
	\chapter*{Abstract}
The work presented in the thesis has been conducted as part of the research and development (R\&D) efforts for the \ac{SciFi} tracker of the LHCb experiment at CERN.  More precisely, new SiPMs are being considered for future upgrades. This report presents a comprehensive characterisation of the \ac{SiPM} array developed at the \ac{FBK} with the near ultraviolet high density (NUV-HD) technology. \\
\\
As a reference and tool of comparison, the actual SiPM used in the SciFi tracker from Hamamatsu, the H2017 is characterised with methods developed in \cite{Girard2018CharacterisationDistributions} that are used as a baseline for the characterisation of the FBK SiPM. The quenching resistors $R_Q$, breakdown voltage $V_{bd}$, correlated noise, gain, and \ac{PDE} are measured. \\
\\
The results from the characterisation showed a higher operation range for the FBK compared to the Hamamatsu.
The FBK detectors showed a correlated noise value significantly lower than the H2017, with values from $3.8\%$ to $5.8\%$ at $\Delta V = 12$ V compared to $9.5\%$ at $\Delta V = 4.4$ V for the H2017. \\

The PDE of the FBK detectors reaches a maximum around \SI{410}{\nano m} of $29\%$, $52\%$, $57\%$ for the \SI{16}{\micro m} , \SI{31}{\micro m} and \SI{42}{\micro m} respectively, at $\Delta V = 12$ V. The Hamamatsu H2017 has at $\lambda \approx 470$ nm a peak PDE of $40\%$ at $\Delta V = 4.4$ V. Adding an epoxy layer on the FBK \SI{31}{\micro m} cancelled the oscillations one the PDE value and reduced it by $10\%$ of the \SI{31}{\micro m} bare die with a peak efficiency of $46\%$ at $\Delta V=12$ V.
The low correlated noise value for the FBK detectors allow to reach a higher PDE than the Hamamatsu.\\

This work has continued the development of methods to characterise the FBK NUV-HD detectors. \\
\\

Keywords: \textit{Silicon Photo Multiplier SiPM, Characterisation, FBK NUV-HD, Quenching resistor, Breakdown Voltage, Correlated Noise, Gain, Photo-Detection Efficiency.}

\end{otherlanguage}

